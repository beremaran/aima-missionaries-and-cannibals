\documentclass[]{report}


% Title Page
\title{Solution for Missionaries and Cannibals Problem using Search Algorithms}
\author{Berke Emrecan Arslan, Zeynep Gördük}


\begin{document}
\maketitle

\begin{abstract}
\end{abstract}

\section{Introduction}
Many transportation scheduling problems are problems of reasoning about actions. Such problems can be formulated as follows. Given a set of space points, an initial distribution of objects in these points and transportation facilities with given capacities; find an optimal sequence of transportations between the space points such that a terminal distribution of objects in these points can be attained without violating a set of given constraints on possible intermediate distribution of objects.

In a more generalized version of this problem, there are $N$ missionaries and $N$ cannibals (where $N \geq 3$) and the boat has a capacity $k$ (where $k \geq 2$). We call this problem the Missionaries and Cannibals Problem.

\section{Problem Formulation}
We shall formulate now the Missionaries and Cannibals problem in a system of productions of the type described in section 2. We start by specifying a simple but straightforward $N$-state language.

The universe $U_0$ of the $N$-state language contains the following basic elements:
\begin{list}{--}{}
	\item $N$ individuals $m_1,m_2,\dots,m_N$ that are missionaries and $N$ individuals $c_1,c_2,\dots,c_N$ that are cannibals,
	\item an object (transportation facility) -- tho boat $b_k$ with a carrying capacity $k$,
	\item two space points $p_L,p_R$ for the left bank and the right bank of the river respectively.
\end{list}

The basic relations between basic elements in $U_0$ are as follows:
\begin{list}{}{}
	\item $at$; this associates an individual or the boat with a space point (example: at $(m_1,p_L)$ asserts the missionary $m_1$ is at the left bank)
	\item $on$; this indicates that an individual is aboard the boat (example: on $(c_1,b_k)$ asserts that the cannibal $c_1$ is on the boat)
\end{list}

A set of expressions, one for each individual and one for the boat (they specify the positions of all the individuals and of the boat) provides a basic description of a situation, i.e. it characterizes an $N$-state. Thus, the initial $N$-state for the Missionaries and Cannibals problem can be written as follows:

\begin{displaymath}
s_0 = at(b_k,p_L),at(m_1,p_L),\dots,at(m_n,p_L),at(c_1,p_L),\dots,at(c_N,p_L)
\end{displaymath}

The terminal $N$-state is attained by substituting $p_R$ for $p_L$ through-out.

\end{document}          
